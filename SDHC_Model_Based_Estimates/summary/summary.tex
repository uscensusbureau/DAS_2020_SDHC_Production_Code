\documentclass{article}

\usepackage{amsmath, amssymb, booktabs}


\begin{document}

Summary of the results of Experiment 5, State and National tables.

\section{Count tables}

Tables PH2, PH3, PH4, PH6, PH7

The noisy measurements, \(Z\), are draws from a discrete Gaussian distribution
with unknown mean, \(Y\), and known variance \(D\), so that
\begin{equation*}
  Z = Y + \varepsilon,
\end{equation*}
where \(\varepsilon \sim DG(Y, D)\).  Our goal is estimation of the true Census
count, \(Y\).  For simplicity, in our modeling work, we replace the discrete
Gaussian distribution with a continuous Gaussian distribution.

Because \(Y\) is a count, we have the logical constraint that \(Y \geq 0\).  We
can incorporate this information into the model through use of a prior
distribution.  We use the improper prior
\begin{equation*}
  g (Y) \propto I \left( Y \geq 0 \right).
\end{equation*}
This results in the posterior distribution
\begin{equation*}
  \pi \left( Y \mid Z \right) \propto \exp \left\{ -\frac{1}{2D} \left( Y - Z
    \right)^2 \right\} I \left( Y \geq 0 \right),
\end{equation*}
which is a truncated normal distribution.  The mean of this distribution has a
closed form representation, but the quantiles need to be estimated numerically.

The advantages of this methodology are
\begin{itemize}
  \item Simplicity
  \item Speed
  \item Does not require additional auxiliary information
  \item No possibility of model misspecification
  \item Produces non-negative estimates
  \item Gives accurate interval estimates
\end{itemize}

The disadvantages of this methodology are
\begin{itemize}
  \item Does not make use of auxiliary information
  \item Predictions are not meaningfully different from noisy measurements when
    the true counts are large
\end{itemize}

% latex table generated in R 4.3.1 by xtable 1.8-4 package
% Wed Jul 12 14:55:59 2023
\begin{table}[ht]
\centering
\begin{tabular}{rrrrrrrrrr}
  \toprule
 & min & max & p\_neg & RMSE & COV & LEN & RMSE\_s & COV\_s & LEN\_s \\ 
  \midrule
PH2\_NM & 1478.00 & 36320973.00 & 0.00 & 158.02 & 90.95 & 399.97 & 107.74 & 88.24 & 399.97 \\ 
  PH2\_PRED & 1481.00 & 36320970.00 & 0.00 & 157.98 & 91.10 & 403.16 & 107.56 & 88.24 & 398.63 \\ 
  PH3\_NM & -64.00 & 9033149.00 & 0.63 & 96.61 & 89.80 & 56.56 & 14.16 & 90.59 & 40.00 \\ 
  PH3\_PRED & 4.00 & 9033149.00 & 0.00 & 96.52 & 91.37 & 55.94 & 14.36 & 94.51 & 38.14 \\ 
  PH4\_NM & -281.00 & 29788110.00 & 0.20 & 139.95 & 90.72 & 399.97 & 150.64 & 85.71 & 399.97 \\ 
  PH4\_PRED & 48.00 & 29788109.00 & 0.00 & 139.39 & 90.72 & 401.90 & 139.45 & 87.01 & 396.91 \\ 
  PH6\_NM & 158.00 & 7830063.00 & 0.00 & 203.56 & 88.42 & 399.96 & 125.88 & 85.19 & 399.96 \\ 
  PH6\_PRED & 181.00 & 7830059.00 & 0.00 & 203.62 & 88.52 & 401.49 & 125.33 & 85.19 & 399.77 \\ 
  PH7\_NM & -38.00 & 36320845.00 & 0.20 & 92.28 & 89.56 & 135.99 & 45.72 & 91.18 & 135.99 \\ 
  PH7\_PRED & 22.00 & 36320841.00 & 0.00 & 92.19 & 90.10 & 138.04 & 44.54 & 91.18 & 135.56 \\ 
   \bottomrule
\end{tabular}
\end{table}




\section{Ratio tables}

Tables PH1, PH5, PH8

The methodology for the ratio tables is similar to that for the count tables.
We extend what was done for the count tables by modeling the numerator and
the denominator of the ratios jointly.  Let the subscripts \(num\) and \(den\)
denote the numerator and denominator, respectively.  We then assume
\begin{equation*}
  Z_{num} \sim N \left( Y_{num}, D_{num} \right)
\end{equation*}
and
\begin{equation*}
  Z_{den} \sim N \left( Y_{den}, D_{den} \right).
\end{equation*}

We also have the logical constrints
\begin{itemize}
  \item \(Y_{den} \geq 1\) (no areas with zero housing units)
  \item Either \(Y_{num} / Y_{den} \geq 1\) if the universe is households (no
    vacant housing units) or \(Y_{num} / Y_{den} \geq 2\) if the universe is
    families
  \item \(Y_{num} / Y_{den} \leq \kappa\), where \(\kappa\) is the truncation
    level (taken here to be 10, based on the configuration file)
\end{itemize}
These constraints can be summarized with an appropriatey defined matrix
\(\boldsymbol{D}\) and vectors \(\boldsymbol{a}\) and \(\boldsymbol{b}\), so
that
\begin{equation*}
  \boldsymbol{a} \leq \boldsymbol{D} \boldsymbol{Y} \leq \boldsymbol{b},
\end{equation*}
where \(\boldsymbol{Y} = \left( Y_{num}, Y_{den} \right)^\intercal\).  For
example, for the household ratio tables, we would set
\begin{equation*}
  \boldsymbol{a} = \begin{bmatrix} 1 \\ 0 \\ 0 \end{bmatrix}, \
  \boldsymbol{b} = \begin{bmatrix} \infty \\ \infty \\ \infty \end{bmatrix}, \
  \boldsymbol{D} = \begin{bmatrix} 0 & 1 \\ 1 & -1 \\ -1 & \kappa \end{bmatrix}
\end{equation*}



We use the improper prior distribution
\begin{equation*}
  g ( \boldsymbol{Y} ) \propto I \left( \boldsymbol{a} \leq \boldsymbol{D}
    \boldsymbol{Y} \leq \boldsymbol{b} \right),
\end{equation*}
which results in a posterior distribution
\begin{equation*}
  \pi \left( \boldsymbol{Y} \mid \boldsymbol{Z} \right) \propto \exp \left\{
    -\frac{1}{2} \left( \boldsymbol{Y} - \boldsymbol{Z} \right)^\intercal
    \boldsymbol{D} \left( \boldsymbol{Y} - \boldsymbol{Z} \right) \right\} I
    \left( \boldsymbol{a} \leq \boldsymbol{D} \boldsymbol{Y} \leq \boldsymbol{b}
    \right),
\end{equation*}
which is a truncated multivariate normal distribution.  We generate samples from
this distribution using the {\tt R} package {\tt tmvmixnrom}.

The advantages of this methodology are
\begin{itemize}
  \item Simplicity
  \item Speed
  \item Does not require additional auxiliary information
  \item Produces ratios that are bounded between 1 (or 2 for family tables) and
    the truncation level
  \item Gives accurate interval estimates for {\em all} ratios
\end{itemize}

The disadvantages of this methodology are
\begin{itemize}
  \item Does not make use of auxiliary information
  \item Predictions are not meaningfully different from noisy measurements when
    both the true numerator and denominator are large
  \item Intervals are not symmetric around the point estimates (although this is
    not unique to this methodology)
\end{itemize}

% latex table generated in R 4.3.1 by xtable 1.8-4 package
% Wed Jul 12 14:55:59 2023
\begin{table}[ht]
\centering
\begin{tabular}{rrrrrrr}
  \toprule
 & min & max & p\_bad & RMSE & COV & LEN \\ 
  \midrule
PH1\_NM & -3.67 & 11.92 & 0.78 & 0.51 & 88.50 & 0.45 \\ 
  PH1\_PRED & 0.17 & 6.32 & 0.00 & 0.25 & 90.26 & 0.26 \\ 
  PH5\_NM & -5.17 & 17.21 & 0.72 & 0.69 & 87.52 & 1.18 \\ 
  PH5\_PRED & 0.50 & 6.21 & 0.00 & 0.20 & 89.48 & 0.34 \\ 
  PH8\_NM & -26.25 & 16.12 & 0.26 & 1.31 & 89.93 & 11.30 \\ 
  PH8\_PRED & 1.44 & 6.06 & 0.00 & 0.23 & 90.59 & 0.25 \\ 
   \bottomrule
\end{tabular}
\end{table}




\end{document}
